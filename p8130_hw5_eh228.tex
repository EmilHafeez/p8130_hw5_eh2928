% Options for packages loaded elsewhere
\PassOptionsToPackage{unicode}{hyperref}
\PassOptionsToPackage{hyphens}{url}
%
\documentclass[
]{article}
\usepackage{lmodern}
\usepackage{amssymb,amsmath}
\usepackage{ifxetex,ifluatex}
\ifnum 0\ifxetex 1\fi\ifluatex 1\fi=0 % if pdftex
  \usepackage[T1]{fontenc}
  \usepackage[utf8]{inputenc}
  \usepackage{textcomp} % provide euro and other symbols
\else % if luatex or xetex
  \usepackage{unicode-math}
  \defaultfontfeatures{Scale=MatchLowercase}
  \defaultfontfeatures[\rmfamily]{Ligatures=TeX,Scale=1}
\fi
% Use upquote if available, for straight quotes in verbatim environments
\IfFileExists{upquote.sty}{\usepackage{upquote}}{}
\IfFileExists{microtype.sty}{% use microtype if available
  \usepackage[]{microtype}
  \UseMicrotypeSet[protrusion]{basicmath} % disable protrusion for tt fonts
}{}
\makeatletter
\@ifundefined{KOMAClassName}{% if non-KOMA class
  \IfFileExists{parskip.sty}{%
    \usepackage{parskip}
  }{% else
    \setlength{\parindent}{0pt}
    \setlength{\parskip}{6pt plus 2pt minus 1pt}}
}{% if KOMA class
  \KOMAoptions{parskip=half}}
\makeatother
\usepackage{xcolor}
\IfFileExists{xurl.sty}{\usepackage{xurl}}{} % add URL line breaks if available
\IfFileExists{bookmark.sty}{\usepackage{bookmark}}{\usepackage{hyperref}}
\hypersetup{
  pdftitle={P8130 Biostatistical Methods Homework 5},
  pdfauthor={Emil Hafeez (eh2928)},
  hidelinks,
  pdfcreator={LaTeX via pandoc}}
\urlstyle{same} % disable monospaced font for URLs
\usepackage[margin=1in]{geometry}
\usepackage{color}
\usepackage{fancyvrb}
\newcommand{\VerbBar}{|}
\newcommand{\VERB}{\Verb[commandchars=\\\{\}]}
\DefineVerbatimEnvironment{Highlighting}{Verbatim}{commandchars=\\\{\}}
% Add ',fontsize=\small' for more characters per line
\usepackage{framed}
\definecolor{shadecolor}{RGB}{248,248,248}
\newenvironment{Shaded}{\begin{snugshade}}{\end{snugshade}}
\newcommand{\AlertTok}[1]{\textcolor[rgb]{0.94,0.16,0.16}{#1}}
\newcommand{\AnnotationTok}[1]{\textcolor[rgb]{0.56,0.35,0.01}{\textbf{\textit{#1}}}}
\newcommand{\AttributeTok}[1]{\textcolor[rgb]{0.77,0.63,0.00}{#1}}
\newcommand{\BaseNTok}[1]{\textcolor[rgb]{0.00,0.00,0.81}{#1}}
\newcommand{\BuiltInTok}[1]{#1}
\newcommand{\CharTok}[1]{\textcolor[rgb]{0.31,0.60,0.02}{#1}}
\newcommand{\CommentTok}[1]{\textcolor[rgb]{0.56,0.35,0.01}{\textit{#1}}}
\newcommand{\CommentVarTok}[1]{\textcolor[rgb]{0.56,0.35,0.01}{\textbf{\textit{#1}}}}
\newcommand{\ConstantTok}[1]{\textcolor[rgb]{0.00,0.00,0.00}{#1}}
\newcommand{\ControlFlowTok}[1]{\textcolor[rgb]{0.13,0.29,0.53}{\textbf{#1}}}
\newcommand{\DataTypeTok}[1]{\textcolor[rgb]{0.13,0.29,0.53}{#1}}
\newcommand{\DecValTok}[1]{\textcolor[rgb]{0.00,0.00,0.81}{#1}}
\newcommand{\DocumentationTok}[1]{\textcolor[rgb]{0.56,0.35,0.01}{\textbf{\textit{#1}}}}
\newcommand{\ErrorTok}[1]{\textcolor[rgb]{0.64,0.00,0.00}{\textbf{#1}}}
\newcommand{\ExtensionTok}[1]{#1}
\newcommand{\FloatTok}[1]{\textcolor[rgb]{0.00,0.00,0.81}{#1}}
\newcommand{\FunctionTok}[1]{\textcolor[rgb]{0.00,0.00,0.00}{#1}}
\newcommand{\ImportTok}[1]{#1}
\newcommand{\InformationTok}[1]{\textcolor[rgb]{0.56,0.35,0.01}{\textbf{\textit{#1}}}}
\newcommand{\KeywordTok}[1]{\textcolor[rgb]{0.13,0.29,0.53}{\textbf{#1}}}
\newcommand{\NormalTok}[1]{#1}
\newcommand{\OperatorTok}[1]{\textcolor[rgb]{0.81,0.36,0.00}{\textbf{#1}}}
\newcommand{\OtherTok}[1]{\textcolor[rgb]{0.56,0.35,0.01}{#1}}
\newcommand{\PreprocessorTok}[1]{\textcolor[rgb]{0.56,0.35,0.01}{\textit{#1}}}
\newcommand{\RegionMarkerTok}[1]{#1}
\newcommand{\SpecialCharTok}[1]{\textcolor[rgb]{0.00,0.00,0.00}{#1}}
\newcommand{\SpecialStringTok}[1]{\textcolor[rgb]{0.31,0.60,0.02}{#1}}
\newcommand{\StringTok}[1]{\textcolor[rgb]{0.31,0.60,0.02}{#1}}
\newcommand{\VariableTok}[1]{\textcolor[rgb]{0.00,0.00,0.00}{#1}}
\newcommand{\VerbatimStringTok}[1]{\textcolor[rgb]{0.31,0.60,0.02}{#1}}
\newcommand{\WarningTok}[1]{\textcolor[rgb]{0.56,0.35,0.01}{\textbf{\textit{#1}}}}
\usepackage{graphicx,grffile}
\makeatletter
\def\maxwidth{\ifdim\Gin@nat@width>\linewidth\linewidth\else\Gin@nat@width\fi}
\def\maxheight{\ifdim\Gin@nat@height>\textheight\textheight\else\Gin@nat@height\fi}
\makeatother
% Scale images if necessary, so that they will not overflow the page
% margins by default, and it is still possible to overwrite the defaults
% using explicit options in \includegraphics[width, height, ...]{}
\setkeys{Gin}{width=\maxwidth,height=\maxheight,keepaspectratio}
% Set default figure placement to htbp
\makeatletter
\def\fps@figure{htbp}
\makeatother
\setlength{\emergencystretch}{3em} % prevent overfull lines
\providecommand{\tightlist}{%
  \setlength{\itemsep}{0pt}\setlength{\parskip}{0pt}}
\setcounter{secnumdepth}{-\maxdimen} % remove section numbering

\title{P8130 Biostatistical Methods Homework 5}
\author{Emil Hafeez (eh2928)}
\date{11/13/2020}

\begin{document}
\maketitle

\hypertarget{problem-1}{%
\section{Problem 1}\label{problem-1}}

\begin{Shaded}
\begin{Highlighting}[]
\CommentTok{#Read the CSV data into a dataframe}
\NormalTok{antibodies_df <-}\StringTok{ }\KeywordTok{read.csv}\NormalTok{(}\StringTok{"./data/Antibodies.csv"}\NormalTok{)}

\CommentTok{# Make AgeCategory, Smell, and Gender appopriate datatypes. Helps to ensure we know all unique values, too.}
\KeywordTok{unique}\NormalTok{(antibodies_df}\OperatorTok{$}\NormalTok{AgeCategory)}
\NormalTok{    antibodies_df <-}\StringTok{ }\NormalTok{antibodies_df }\OperatorTok\StringTok{ }
\StringTok{        }\KeywordTok{mutate}\NormalTok{(}\DataTypeTok{AgeCategory =} \KeywordTok{factor}\NormalTok{(AgeCategory, }\DataTypeTok{labels =} \KeywordTok{c}\NormalTok{(}\StringTok{"18-30"}\NormalTok{, }\StringTok{"31-50"}\NormalTok{, }\StringTok{"51+"}\NormalTok{) ))}
\KeywordTok{unique}\NormalTok{(antibodies_df}\OperatorTok{$}\NormalTok{Smell)}
\NormalTok{    antibodies_df <-}\StringTok{ }\NormalTok{antibodies_df }\OperatorTok\StringTok{ }
\StringTok{        }\KeywordTok{mutate}\NormalTok{(}\DataTypeTok{Smell =} \KeywordTok{factor}\NormalTok{(Smell, }\DataTypeTok{levels =} \KeywordTok{c}\NormalTok{(}\StringTok{"Normal"}\NormalTok{,}\StringTok{"Altered"}\NormalTok{, }\StringTok{"Unanswered/Others"}\NormalTok{)))}
\NormalTok{    antibodies_df =}\StringTok{ }\NormalTok{antibodies_df }\OperatorTok\StringTok{ }\KeywordTok{filter}\NormalTok{(Smell }\OperatorTok{!=}\StringTok{ "Unanswered/Others"}\NormalTok{)}
\KeywordTok{unique}\NormalTok{(antibodies_df}\OperatorTok{$}\NormalTok{Gender)}
\NormalTok{    antibodies_df <-}\StringTok{ }\NormalTok{antibodies_df }\OperatorTok\StringTok{ }
\StringTok{        }\KeywordTok{mutate}\NormalTok{(}\DataTypeTok{Gender =} \KeywordTok{factor}\NormalTok{(Gender, }\DataTypeTok{levels =} \KeywordTok{c}\NormalTok{(}\StringTok{"Male"}\NormalTok{,}\StringTok{"Female"}\NormalTok{)))}
\end{Highlighting}
\end{Shaded}

In order to assess the difference in IgM levels between the two smell
factor groups (Normal vs Altered) given non-normal distributions, we opt
for a non-parametric test called the Wilcoxon Rank-Sum test. It's the
nonparametric equivalent of the two-sample independent t-test, and
examines if the medians of the two populations are equal versus not
equal:

\(H_0 =\) the medians of the two groups' IgM levels are equal, and
\(H_A =\) the medians of the two groups' IgM levels are not equal. The
decision rule is that we reject \(H_0 =\) if \(T > z_{1-(\alpha/2)}\).

The test statistic is computed, with a continuity correction, one of two
ways. We first combine the data from the two groups, order the values
from lowest to highest, assign ranks to the individual values (1 to n),
and if ties, assign the average rank. Then, select a group and compute
the sum of ranks \(T_1\) for the first group, and then use the
appropriate test statistic formula.

With no ties (referring to two equally ranked values once the values are
listed), the test statistic is

\(T=\frac{\left|T_{1}-\frac{n_{1}\left(n_{1}+n_{2}+1\right)}{2}\right|-\frac{1}{2}}{\sqrt{\left(n_{1} n_{2} / 12\right)\left(n_{1}+n_{2}+1\right)}}\)

and with ties, the test statistic is

\(T=\frac{\left|T_{1}-\frac{n_{1}\left(n_{1}+n_{2}+1\right)}{2}\right|-\frac{1}{2}}{\sqrt{\left(n_{1} n_{2} / 12\right)\left[\left(n_{1}+n_{2}+1\right)-\sum_{i=1}^{g} t_{i}\left(t_{i}^{2}-1\right) /\left(n_{1}+n_{2}\right)\left(n_{1}+n_{2}-1\right)\right]}}\)

where \(t_i\) refers to the number of observations with the same
absolute value in the \(i^{th}\) group and \(g\) is the number of tied
groups.

In our case, the test statistic calculated by R is slightly different,
since it does not by default add the \(n_1(n_1+1)/2\) term (and is
denoted by W).

The p-value under the normal approximation, with \(n_1\) and \(n_2\)
\(\geq 10\) is described by \(2 * [1 -\Phi(T)]\).

\begin{Shaded}
\begin{Highlighting}[]
\NormalTok{antibodies_df2 =}
\StringTok{  }\NormalTok{antibodies_df }\OperatorTok\StringTok{ }
\StringTok{  }\KeywordTok{pivot_wider}\NormalTok{(}
        \DataTypeTok{names_from =}\NormalTok{ Smell,}
        \DataTypeTok{values_from =}\NormalTok{ Antibody_IgM}
\NormalTok{    )}

\KeywordTok{wilcox.test}\NormalTok{(antibodies_df2}\OperatorTok{$}\NormalTok{Normal, antibodies_df2}\OperatorTok{$}\NormalTok{Altered, }\DataTypeTok{mu =} \DecValTok{0}\NormalTok{)}
\end{Highlighting}
\end{Shaded}

\begin{verbatim}
## 
##  Wilcoxon rank sum test with continuity correction
## 
## data:  antibodies_df2$Normal and antibodies_df2$Altered
## W = 5836, p-value = 0.01406
## alternative hypothesis: true location shift is not equal to 0
\end{verbatim}

In context, and ignoring missing values and the unanswered smell
category, we find evidence to reject the null hypothesis and conclude
that the true location shift between the Normal and Altered smell
categories is not equal to zero (in other words, the median IgM values
are different for the two groups).

\hypertarget{problem-2}{%
\section{Problem 2}\label{problem-2}}

Let's come back to this.

\hypertarget{problem-3}{%
\section{Problem 3}\label{problem-3}}

\hypertarget{problem-3-part-1}{%
\subsection{Problem 3 Part 1}\label{problem-3-part-1}}

Load the data and plot it.

\begin{Shaded}
\begin{Highlighting}[]
\NormalTok{gpa_df <-}\StringTok{ }\KeywordTok{read.csv}\NormalTok{(}\StringTok{"./data/GPA.csv"}\NormalTok{)}

\KeywordTok{plot}\NormalTok{(gpa_df}\OperatorTok{$}\NormalTok{ACT, gpa_df}\OperatorTok{$}\NormalTok{GPA)}
\end{Highlighting}
\end{Shaded}

\includegraphics{p8130_hw5_eh228_files/figure-latex/unnamed-chunk-2-1.pdf}
Test whether a linear association exists between ACT score (x) and GPA
at the end of freshman year (Y).

Let \(\alpha = 0.05\) and let \(\beta_1\) represent the true slope to be
estimated.

The null hypothesis is \(H_0: \beta_1 = \beta_{10}\) where
\(\beta_{10} = 0\). The alternative hypothesis is
\(H_A: \beta_1 \neq \beta_{10}\). In context, testing
\(H_0:\beta_1 = 0\) examines whether a student's GPA at the end of
freshman year can be predicted from the ACT test score.

The test statistic follows the t distribution with n-2 degrees of
freedom, such that

\(t=\frac{\widehat{\beta_{1}}-\beta_{10}}{s e\left(\widehat{\beta_{1}}\right)} \sim t_{n-2}, \text { under } H_{0} = \frac{0.03883 - 0}{0.01277} = 3.040\)
using degrees of freedom = \(n = 120, df = n-2\)

The corresponding critical value is fixed by \(t_{n-2, 1-(\alpha/2)}\)
and the decision rule is that we reject \(H_0\) if
\(|t| > t_{n-2, 1-(\alpha/2)}\) and fail to reject \(H_0\) if
\(|t| \leq t_{n-2, 1-(\alpha/2)}\). As such, the critical value is
\(t_{118, 0.975} = 1.980272\).

Therefore, \(|t| > t_{118, 0.975}\) using the 5\% significance level, we
find evidence to reject the null hypothesis and conclude that there is a
significant linear association between students' ACT scores and GPA at
the end of freshman year.

\begin{Shaded}
\begin{Highlighting}[]
\NormalTok{reg_admit<-}\KeywordTok{lm}\NormalTok{(gpa_df}\OperatorTok{$}\NormalTok{GPA}\OperatorTok{~}\NormalTok{gpa_df}\OperatorTok{$}\NormalTok{ACT)}

\CommentTok{# Summarize regression}
\KeywordTok{summary}\NormalTok{(reg_admit)}
\end{Highlighting}
\end{Shaded}

\begin{verbatim}
## 
## Call:
## lm(formula = gpa_df$GPA ~ gpa_df$ACT)
## 
## Residuals:
##      Min       1Q   Median       3Q      Max 
## -2.74004 -0.33827  0.04062  0.44064  1.22737 
## 
## Coefficients:
##             Estimate Std. Error t value Pr(>|t|)    
## (Intercept)  2.11405    0.32089   6.588  1.3e-09 ***
## gpa_df$ACT   0.03883    0.01277   3.040  0.00292 ** 
## ---
## Signif. codes:  0 '***' 0.001 '**' 0.01 '*' 0.05 '.' 0.1 ' ' 1
## 
## Residual standard error: 0.6231 on 118 degrees of freedom
## Multiple R-squared:  0.07262,    Adjusted R-squared:  0.06476 
## F-statistic:  9.24 on 1 and 118 DF,  p-value: 0.002917
\end{verbatim}

\begin{Shaded}
\begin{Highlighting}[]
\KeywordTok{tidy}\NormalTok{(reg_admit)}
\end{Highlighting}
\end{Shaded}

\begin{verbatim}
## # A tibble: 2 x 5
##   term        estimate std.error statistic       p.value
##   <chr>          <dbl>     <dbl>     <dbl>         <dbl>
## 1 (Intercept)   2.11      0.321       6.59 0.00000000130
## 2 gpa_df$ACT    0.0388    0.0128      3.04 0.00292
\end{verbatim}

\begin{Shaded}
\begin{Highlighting}[]
\KeywordTok{glance}\NormalTok{(reg_admit)}
\end{Highlighting}
\end{Shaded}

\begin{verbatim}
## # A tibble: 1 x 12
##   r.squared adj.r.squared sigma statistic p.value    df logLik   AIC   BIC
##       <dbl>         <dbl> <dbl>     <dbl>   <dbl> <dbl>  <dbl> <dbl> <dbl>
## 1    0.0726        0.0648 0.623      9.24 0.00292     1  -113.  231.  239.
## # ... with 3 more variables: deviance <dbl>, df.residual <int>, nobs <int>
\end{verbatim}

\begin{Shaded}
\begin{Highlighting}[]
\CommentTok{# Regression objects}
\KeywordTok{names}\NormalTok{(reg_admit)}
\end{Highlighting}
\end{Shaded}

\begin{verbatim}
##  [1] "coefficients"  "residuals"     "effects"       "rank"         
##  [5] "fitted.values" "assign"        "qr"            "df.residual"  
##  [9] "xlevels"       "call"          "terms"         "model"
\end{verbatim}

\begin{Shaded}
\begin{Highlighting}[]
\CommentTok{# Get fitted.values}
\NormalTok{reg_admit}\OperatorTok{$}\NormalTok{fitted.values}
\end{Highlighting}
\end{Shaded}

\begin{verbatim}
##        1        2        3        4        5        6        7        8 
## 2.929419 2.657629 3.201209 2.968246 2.929419 3.317690 3.356517 3.162382 
##        9       10       11       12       13       14       15       16 
## 3.240036 3.123555 3.045900 3.278863 3.045900 3.045900 3.395344 3.162382 
##       17       18       19       20       21       22       23       24 
## 3.084727 3.317690 3.084727 2.890592 3.045900 2.929419 3.201209 3.162382 
##       25       26       27       28       29       30       31       32 
## 3.201209 3.123555 3.201209 2.968246 3.123555 2.929419 3.084727 2.735283 
##       33       34       35       36       37       38       39       40 
## 3.201209 3.123555 2.968246 3.045900 2.929419 3.278863 3.162382 3.123555 
##       41       42       43       44       45       46       47       48 
## 3.123555 3.278863 3.045900 3.123555 3.240036 3.045900 3.317690 2.696456 
##       49       50       51       52       53       54       55       56 
## 2.851765 2.812938 3.162382 2.735283 3.162382 3.123555 3.045900 3.278863 
##       57       58       59       60       61       62       63       64 
## 2.929419 2.890592 3.278863 3.240036 3.084727 3.007073 3.084727 3.007073 
##       65       66       67       68       69       70       71       72 
## 3.278863 2.929419 3.045900 3.356517 2.812938 3.007073 2.890592 3.007073 
##       73       74       75       76       77       78       79       80 
## 2.812938 2.812938 3.240036 2.890592 3.007073 3.123555 3.201209 3.434172 
##       81       82       83       84       85       86       87       88 
## 2.890592 2.890592 3.123555 3.356517 3.084727 3.162382 3.162382 3.240036 
##       89       90       91       92       93       94       95       96 
## 2.851765 2.929419 3.045900 3.162382 3.084727 2.812938 3.240036 3.045900 
##       97       98       99      100      101      102      103      104 
## 3.162382 2.929419 2.851765 2.812938 3.084727 2.812938 2.890592 3.356517 
##      105      106      107      108      109      110      111      112 
## 3.045900 3.472999 3.084727 3.201209 3.201209 3.084727 2.968246 3.278863 
##      113      114      115      116      117      118      119      120 
## 2.890592 2.890592 3.317690 2.890592 3.240036 3.201209 2.735283 3.201209
\end{verbatim}

\begin{Shaded}
\begin{Highlighting}[]
\CommentTok{# Scatterplot and regression line overlaid}
\KeywordTok{plot}\NormalTok{(gpa_df}\OperatorTok{$}\NormalTok{ACT, gpa_df}\OperatorTok{$}\NormalTok{GPA)}
\KeywordTok{abline}\NormalTok{(reg_admit,}\DataTypeTok{lwd=}\DecValTok{2}\NormalTok{,}\DataTypeTok{col=}\DecValTok{2}\NormalTok{)}
\end{Highlighting}
\end{Shaded}

\includegraphics{p8130_hw5_eh228_files/figure-latex/unnamed-chunk-3-1.pdf}

\hypertarget{problem-3-part-2}{%
\subsection{Problem 3 Part 2}\label{problem-3-part-2}}

The basic regression model follows the form
\(Y_i = \beta_0 + \beta_1X_i + \epsilon_i\) and the estimated regression
model is given by
\(\widehat{Y} = \widehat{\beta_0} + \widehat{\beta_1}X_i, i = 1,2,3...n\).
In our context, this means that the estimated GPA value is equal to the
sum of the intercept \(\beta_0\) and the estimated beta one times the
student's ACT score, such that
\(\widehat{GPA} = 2.11405 + 0.03883 \cdot ACT\).

\hypertarget{problem-3-part-3}{%
\subsection{Problem 3 Part 3}\label{problem-3-part-3}}

Obtain a 95\% confidence interval for β1. Interpret your confidence
interval. Does it include zero? Why might the director of admissions be
interested in whether the confidence interval includes zero? (2.5p)

The 95\% confidence interval for the true slope

\end{document}
